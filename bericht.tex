\section{Bericht}

\subsection{Einleitung}
Ziel der Übung war es den Winkel $\alpha$ als Funktion, abhängig vom Winkel $\beta$ zu bestimmen und einen Graphen der Funktion $\alpha = f(\beta)$ darzustellen. Der Graph sollte im Bereich $\beta = -300$ Grad bis $\beta = +300$ Grad gezeichnet werden. \\
Gegeben war uns eine Einführung in 


%Um dies zu schaffen haben wir mit Hilfe der gegebenen Formeln und Abbildungen zuerst die Funktion aufzustellen und mit %dieser ein Scilab Script geschrieben welches den Graphen abbildet.

%Zuerst haben wir die Formel Beta + Gamma1 + Gamma = pi nach Gamma1 umgestellt und mit Hilfe der bereits gegebenen Werte für Beta und Gamma Gamma1 berechnet.
 
%Danach konnten wir mit Hilfe der Formel c1^2 = a1^2 + b1^2 – 2a1b1 * cos Gamma1 die dritte Seite c1 des ersten Dreiecks berechnen.
 
%Nachdem wir dann die Formel b2^2 = a2^2 + c2^2 -2*a2c1 * cos Beta2 umgestellt hatten konnten wir Beta2 herausfinden.
 
%Um Alpha zu berechnen benötigten wir noch den Winkel Alpha1. Dieses konnten wir mit der Formel a1^2 = c1^2 + b1^2 – 2*c1b1 * cos Alpha1.
 
%Zum Schluss haben wir die Formel Alpha + Delta + Gamma = Alpha1 + Beta1 nach Alpha umgestellt und ausgerechnet.
 
%Diese Rechenschritte haben wir dann in einen Scilab Skript verwendet um den Graphen zu zeichnen. Mit Hilfe einer If-Abfrage konnten wir das Skript leicht abändern um eine Vergrößerung des Bereichs von Beta von ~40° auf ~300° zu vergrößern. Dies war nötig da man verschiedene Fälle betrachten muss wenn der Winkel Beta kleiner als -Gamma wird.