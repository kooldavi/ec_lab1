\section{Bericht}

\subsection{Einleitung}
Ziel der Übung ist es den Winkel $\alpha$ als Funktion, abhängig vom Winkel $\beta$ zu bestimmen und einen Graphen der Funktion $\alpha = f(\beta)$ darzustellen. Der Graph soll im Bereich $\beta = -300$ Grad bis $\beta = +300$ Grad gezeichnet werden. \\
Gegeben ist uns eine Versuchsbeschreibung mit detaillierten Grafiken zur Geometrie.

\subsection{Durchführung}
Um die gestellt Aufgabe zu schaffen, haben wir mit Hilfe der gegebenen Formeln und Abbildungen zuerst die Funktion aufgestellt und mit dieser ein Scilab-Skript geschrieben, welches den Graphen abbildet.\\
Zuerst haben wir die Formel $\beta + \gamma_1 + \gamma = \pi$ nach $\gamma_1$ umgestellt. Da der Wert für $\gamma = arctan(\frac{h}{d})$ konstant und $\beta$ der Winkel des Motorarms ist ($\beta$ ist zwischen $-300$ Grad und $+300$ Grad), können wir den Wert für $\gamma_1$ und ein bestimmtes $\beta$ direkt errechnen. Wir erhalten also ein Dreieck, in dem uns die Länge der beiden Katheten ($a_1$, $b_1$) und der Wert des eingeschlossenen Winkels ($\gamma_1$) bekannt sind. \\
Nach dem Kosinussatz können wir nun die Hypotenuse berechnen:\\ $c_1^2 = a_1^2 + b_1^2 -2a_1b_1cos(\gamma_1) \Rightarrow c_1 = \sqrt{a_1^2 + b_1^2 -2a_1b_1cos(\gamma_1)}$

 
%Nachdem wir dann die Formel b2^2 = a2^2 + c2^2 -2*a2c1 * cos Beta2 umgestellt hatten konnten wir Beta2 herausfinden.
 
%Um Alpha zu berechnen benötigten wir noch den Winkel Alpha1. Dieses konnten wir mit der Formel a1^2 = c1^2 + b1^2 – 2*c1b1 * cos Alpha1.
 
%Zum Schluss haben wir die Formel Alpha + Delta + Gamma = Alpha1 + Beta1 nach Alpha umgestellt und ausgerechnet.
 
%Diese Rechenschritte haben wir dann in einen Scilab Skript verwendet um den Graphen zu zeichnen. Mit Hilfe einer If-Abfrage konnten wir das Skript leicht abändern um eine Vergrößerung des Bereichs von Beta von ~40° auf ~300° zu vergrößern. Dies war nötig da man verschiedene Fälle betrachten muss wenn der Winkel Beta kleiner als -Gamma wird.